\section{Przykład użycia RBM}
    \paragraph{}
	Nasz projekt ograniczonej maszyny boltzmanna o koncepcje biblioteki książek.
	Użytkownik podaje swoje preferencje w sposób binarny - książka podoba mu się lub nie.
	Przed tym następuje proces trenowania maszyny oparty na preferencjach innych użytkowników biblioteki.
	Wynik jest następie przepuszczny w dodatnią fazę \textit{kontrastowej rozbieżności},
	aby znaleść wspólne cechy pozycji które podał użytkownik (obliczenie ukrytych neuronów)
	Następnie dane przepuszczamy przez ujemną fazę \textit{kontrastowej rozbieżności},
	aby znaleść inne pozycje które mogą spodobać się użytkownikowi.
    \paragraph{}
	Obracowana maszyna Boltzmanna zawiera 21 widocznych neuronów, które oznaczają książkowe preferencje użytkownika oraz 4 ukryte, które oznaczają  4
	różnych autorów książek. Spodziewamy się w takim razie wyników, które będą zbliżone do preferencji użytkownika na podstawie autorów książek, które mu się
	podobją. Nie jest to jednak reguła i jak pokazuje praktyka nie zawsze dostaniemy perfekcyjne wyniki.