\section{Wstęp}
    \paragraph{Ograniczona maszyna boltzmanna}
    to sieć neuronowa oparta w pewnym stopniu na prawdopodobieństwie, składająca się z 2 warstw;
    \textit{warstwy widocznej}, której wartości są znane i ustawiamy je samodzielnie oraz z \textit{warstwy ukrytej}
    czyli wartości których nie znamy i próbujemy nauczyć tą sieć. Dodatkowo przy każdej
    aktywacji takiego neuronu dodawana jest \textit{jednostka odchylenia}, która jest
    stała dla każdego wejścia.

    \paragraph{}
        Ważną charakterystyką ograniczonej maszyny boltzmanna jest brak komunikacji pomiędzy węzłami na tym
        samym poziomie, każdy węzeł działa niezależnie od sąsiadów, węzeł z warstwy widocznej
        przekazuje swój wynik do każdego węzła z warstwy ukrytej. Pozwala to uniknąć zaburzeń neuronów związanych
        ze klasyfikacją danych wejściowych z innymi danymi wejściowymi. Dodatkowo możemy przez to używać bardziej
        zaawansowanych algorytmów uczenia takich jak rozbieżność kontrastowa.
    \paragraph{}
	    Ograniczoną maszynę Boltzmanna, jak każdą inną sieć neuronową możemy podzielić na kilka etapów
        obliczeniowych. Pierwszym z nich jest znalezienie macierzy wag - trenowanie sieci neuronowej.
        Drugim etapem jest testowanie na postawie innych danych oraz znalezienie odpowiednich cech jakie
        wyróżnił wytrenowany przez nas algorytm. Ograniczona maszyna Boltzmanna pozwala nam równieć szukać
        widocznych neuronów (danych które zwykle wprowadza użytkownik) posiadając tylko neurony ukryte.